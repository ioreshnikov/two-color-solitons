\documentclass[a4paper, 11pt]{article}
\pdfoutput=1


\usepackage{amsmath}
\usepackage{amssymb}
\usepackage{authblk}
\usepackage[utf8]{inputenc}


\newcommand{\abs}[1]{\left| #1 \right|}
\renewcommand{\theequation}{S.\arabic{equation}}


\begin{document}

\title{Supplementary Material}

\author[1]{Ivan Oreshnikov}
\author[2,3,4]{Oliver Melchert}
\author[2,3]{Stephanie Willms}
\author[3]{Surajit Bose}
\author[2,3]{Ihar Babushkin}
\author[2,3,4]{Ayhan Demircan}
\author[5]{Alexey Yulin}

\affil[1]{Max Planck Institute for Intelligent Systems, Max-Planck-Ring 4, 72076 T\"ubingen, Germany}
\affil[2]{Cluster of Exellence PhoenixD, Welfengarten 1, 30167 Hannover, Germany}
\affil[3]{Institute of Quantum Optics, Leibniz Universit\"at Hannover, Welfengarten 1, 30167 Hannover, Germany}
\affil[4]{Hannover Centre for Optical Technologies, Neinburger Strasse 17, 30167 Hannover, Germany}
\affil[5]{Department of Nanophotonics and Metamaterials, ITMO University, Kronverskiy pr. 49, 19701 St.~Petersburg, Russia}

\date{\today}

\maketitle


\section*{Small internal oscillations of the soliton}

When analyzing the nonlinear scattering near an oscillatory mode we used an expression for the oscillation frequency. In this section we will derive this expression.

Let us return to equation (3) for coupled solitons. We can re-normalize the equations by performing the following transformation
\begin{equation*}
  U_{n} \to \gamma_{n}^{1/2} e^{i \beta_{n} z} \cdot U_{n},
\end{equation*}
which will make the equations symmetric
\begin{equation*}
  i \partial_{z} U_{n}
    - \frac{1}{2} \beta''_{n} \partial_{t}^{2} U_{n}
    + \gamma_{n}^{2} \abs{U_{n}}^{2} U_{n}
    + 2 \gamma_{n} \gamma_{m} \abs{U_{m}}^{2} U_{n} = 0.
\end{equation*}
This in turn allows us to recognize the modified couple of equations as Euler-Lagrange equations for Lagrangian
\begin{equation*}
  \int \limits_{-\infty}^{+\infty}
    \mathcal{L}(U_{1}, \partial_{z} U_{1}, \partial_{t} U_{1}, \ldots)
    \, dt,
\end{equation*}
where Lagrangian density $\mathcal{L}$ is defined as a sum of three components $\mathcal{L} = \mathcal{L}_{1} + \mathcal{L}_{2} + \mathcal{L}_{\text{int}}$, with $L_{n}$ being a single-soliton Lagrangian density
\begin{equation}
  \label{eq:SingleSolitonLagrangianDensity}
  \mathcal{L}_{n} =
  \frac{i}{2} \left(
    \partial_{z} U_{n} \cdot U_{n}^{*} -
    \partial_{z} U_{n}^{*} \cdot U_{n}
  \right)
  + \frac{1}{2} \beta''_{n} \, \partial_{t} U_{n} \partial_{t} U_{n}^{*}
  + \frac{1}{2} \gamma_{n}^{2} \, \abs{U_{n}}^{4},
\end{equation}
and $\mathcal{L}_{\text{int}}$ being the interaction term
\begin{equation}
  \label{eq:InteractionLagrangianDensity}
  \mathcal{L}_{\text{int}} =
    2 \gamma_{1} \gamma_{2}
    \abs{U_{1}}^{2} \abs{U_{2}}^{2}.
\end{equation}
Let us assume that the soliton components $U_{n}$ can be described by the following generic \textit{ansatz}
\begin{equation}
  \label{eq:SolitonAnsatz}
  U_{n}(z, t) = A_{n}(z) S\left(
    \frac{t - t_{n}(z)}{\sigma_{n}(z)}
  \right) \exp\left(
    - i \Omega_{n}(z) t + i \phi_{n}(z)
  \right).
\end{equation}
In here $A_{n}$ is the amplitude of the pulse, $t_{n}$ is the central position, $\sigma_{n}$ is the pulse width, $\Omega_{n}$ is the frequency detuning, $\phi_{n}$ is the phase, and $S(x)$ is function that defines the envelope shape. At the moment we will not specify the concrete form of $S(x)$, but will assume that it is an even function.

Before we continue let us stress one important thing: this ansatz cannot express all the possible internal oscillations of the soliton. One obvious example, as it was noted in the text, is the case of the pulse-width oscillation. In order to capture this dynamics, we need to add frequency chirp to the ansatz.

Substituting \eqref{eq:SolitonAnsatz} into \eqref{eq:SingleSolitonLagrangianDensity} and \eqref{eq:InteractionLagrangianDensity} and integrating over $t$ we arrive at the expressions for the averaged Lagrangians
\begin{gather}
  L_{n} =
    I_{1} \, t_{n} \sigma_{n} A_{n}^{2} \frac{d \Omega_{n}}{d z} +
    I_{1} \, \sigma_{n} A_{n}^{2} \frac{d \phi_{n}}{d z} +
    I_{2} \, \frac{\beta''_{n}}{2} \frac{A_{n}^{2}}{\sigma_{n}} \nonumber \\
    + I_{1} \, \frac{\beta''_{n}}{2} \sigma_{n} A_{n}^{2} \Omega_{n}^{2} +
    I_{3} \, \frac{\gamma_{n}^{2}}{2} \sigma_{n} A_{n}^{4} \\
  L_{\text{int}} =
    2 \, \gamma_{1} \gamma_{2} \, A_{1}^{2} A_{2}^{2} \,
    I_{\text{int}}(\sigma_{1}, \sigma_{2}, t_{1}, t_{2}),
\end{gather}
where the following integrals have been defined
\begin{align*}
  I_{1} =
    \int \limits_{-\infty}^{+\infty}
    S^{2}(x) \, dx &&
  I_{2} =
    \int \limits_{-\infty}^{+\infty}
    (S'(x))^{2} \, dx \\
  I_{3} =
    \int \limits_{-\infty}^{+\infty}
    S^{4}(x) \, dx &&
  I_{\text{int}} =
    \int \limits_{-\infty}^{+\infty}
    S^{2}\left(
      \frac{t - t_{1}}{\sigma_{1}}
    \right)
    S^{2}\left(
      \frac{t - t_{2}}{\sigma_{2}}
    \right) \, dt
\end{align*}
Due to the time invariance in the problem, $I_{\text{int}}$ depends only on the difference between $t_{1}$ and $t_{2}$
\begin{equation*}
  I_{\text{int}} = I_{\text{int}}(t_{1} - t_{2}, \sigma_{1}, \sigma_{2}),
\end{equation*}
and it is an even function of that difference.

The averaged Lagrangian $L = L_{1} + L_{2} + L_{\text{int}}$ is now a function defined in terms of soliton parameters \{\,$A_{n}$, $\sigma_{n}$, $t_{n}$, $\Omega_{n}$, $\phi_{n}$\,\} and only them. Therefore, the Euler-Lagrange equations for the new Lagrangian have to be defined in terms of variations over the soliton parameters
\begin{equation*}
  \frac{\delta L}{\delta P_{n}} =
    \frac{\partial{L}}{{\partial P_{n}}} -
    \frac{d}{dz} \frac{\partial L}{\partial \dot P_{n}} = 0,
\end{equation*}
where $P_{n}$ stands for either $A_{n}$, $\sigma_{n}$, $t_{n}$, $\Omega_{n}$ or $\phi_{n}$. The latter case --- variation with respect to the phase $\phi_{n}$ --- immediately yields the conservation of mass
\begin{equation}
  \label{eq:ConservationOfMass}
  N_{n} = \sigma_{n}(z) A_{n}^{2}(z) = const.
\end{equation}

Variation with respect to the detuning $\Omega_{n}$ fixes the group velocity of individual solitons
\begin{equation}
  \label{eq:SolitonPosition}
  \frac{d t_{n}}{dz} = \beta''_{n} \Omega_{n}(z).
\end{equation}

Variation with respect to the soliton position $t_{n}$ gives us an equation for the frequency
\begin{equation}
  \label{eq:SolitonFrequency}
  \frac{d \Omega_{n}}{dz} =
    2 \cdot \frac{N_{m} \gamma_{1} \gamma_{2}}
           {I_{1} \sigma_{1}(z) \sigma_{2}(z)} \cdot
    \frac{\partial I_{\text{int}}}{\partial t_{n}}.
\end{equation}
The symmetry in the overlap integral $I_{\text{int}}$ with respect to the soliton positions $t_{1}$ and $t_{2}$ leads to conservation of momentum
\begin{equation}
  \label{eq:ConservationOfMomentum}
  N_{1} \Omega_{1}(z) + N_{2} \Omega_{2}(z) = const.
\end{equation}

Finally, the difference between the variations with respect to $A_{n}$ and $\sigma_{n}$ gives us
\begin{equation}
  \label{eq:SolitonWidth}
  I_{2} \beta''_{n} +
  \frac{I_{3} \gamma_{n}}{2} N_{n} \sigma_{n}(z)
  + 2 N_{m} \gamma_{1} \gamma_{2} \frac{\sigma_{n}(z)}{\sigma_{m}(z)} \left(
    I_{\text{int}} + \sigma_{n} \frac{\partial I_{\text{int}}}{\sigma_{n}}
  \right) = 0.
\end{equation}
The very last equation --- omitted here --- is the evolution equation for the phase $\phi_{n}$. The right-hand's side of the equation is quite complicated, but since the phase does not occur anywhere in \eqref{eq:SolitonPosition}, \eqref{eq:SolitonFrequency} or \eqref{eq:SolitonWidth}, it is not important for the remaining analysis.

Let us switch from the the individual soliton positions to the mean position and the relative delay instead
\begin{align*}
  t_{0} = \frac{1}{2} \left(
    t_{1} + t_{2}
  \right) &&
  \Delta t = t_{1} - t_{2}
\end{align*}
Equation for the relative delay $\Delta t$
\begin{equation}
  \label{eq:RelativeDelay}
  \frac{d \Delta t}{dz} =
    \beta''_{1} \Omega_{1}(z) +
    \beta''_{2} \Omega_{2}(z)
\end{equation}
and equations \eqref{eq:SolitonFrequency} and \eqref{eq:SolitonWidth} form a closed system, with equations for $d \Delta t / dz$, $d \Omega_{n} / dz$ acting as equations of motion and equations \eqref{eq:SolitonWidth} fixing the widths $\sigma_{n}(z)$ as functions of $\Delta t$. By differentiating \eqref{eq:RelativeDelay} one more time and using \eqref{eq:SolitonFrequency} we get
\begin{equation*}
  \frac{d^{2} \Delta t}{dz^{2}} +
  2 \frac{
    \gamma_{1} \gamma_{2} \left(
      \beta''_{1} N_{1} + \beta''_{2} N_{2}
    \right)
  }{
    I_{1} \sigma_{1}(\Delta t) \sigma_{2}(\Delta t)
  } \frac{\partial}{\partial \Delta t}
  I_{\text{int}} (\Delta t, \sigma_{1}, \sigma_{2}) = 0
\end{equation*}
To transform this into a harmonic oscillator equation we need to linearize the second term around the equilibrium point $\Delta t = 0$. Since $I_{\text{int}}$ is an even function, the derivative $\partial I_{\text{int}} / \partial \Delta {t}$ is odd and it vanishes at $\Delta t = 0$. This means we can ignore $\Delta t$ dependency in $\sigma_{1}$ and $\sigma_{2}$ --- only the term proportional to $\partial^{2} I_{\text{int}} / \partial \Delta t^{2}$ will survive. Thus we finally arrive at
\begin{equation*}
  \frac{d^{2} \Delta t}{dz^{2}} +
  K_{0}^{2} \Delta t = 0,
\end{equation*}
where the resonance frequency $K_{0}$ is
\begin{equation}
  \label{eq:ResonanceFrequency}
  K_{0}^{2} = 2 \frac{
    \gamma_{1} \gamma_{2} \left(
      \beta''_{1} N_{1} + \beta''_{2} N_{2}
    \right)
  }{
    I_{1} \sigma_{1}(0) \sigma_{2}(0)
  } I_{\text{int}}''(0; \sigma_{1}(0), \sigma_{2}(0)).
\end{equation}

For a more concrete estimate let us finally consider a Gaussian envelope, i.e.~let us set $S(x) = \exp(-x^{2})$. Such a choice of the envelope shape fixes the integrals $I_{1} = \sqrt{\pi / 2}$ and
\begin{equation*}
  I_{\text{int}}(\Delta t, \sigma_{1}, \sigma_{2}) =
  \sqrt{ \frac{\pi}{2} }
  \frac{
    \sigma_{1} \sigma_{2}
  }{
    \sqrt{
      \left(
        \sigma_{1}^{2} + \sigma_{2}^{2}
      \right)
    }
  }
  \cdot \exp \left(
    \frac{
      -2 \Delta t^{2}
    }{
      \sigma_{1}^{2} + \sigma_{2}^{2}
    }
  \right),
\end{equation*}
which finally gives us the following expression for the resonance frequency
\begin{equation}
  \label{eq:ResonanceFrequencyGaussian}
  K_{0}^{2} =
    - \frac{
      8 \, \gamma(\omega_{1}) \gamma(\omega_{2})
    }{
      \left(
        \sigma_{1}^{2} + \sigma_{2}^{2}
      \right)^{3/2}
    } \cdot \left(
      \beta''(\omega_{1}) \sigma_{1} A_{1}^{2} +
      \beta''(\omega_{2}) \sigma_{2} A_{2}^{2}
    \right).
\end{equation}

\end{document}

%%% Local Variables:
%%% mode: latex
%%% TeX-master: t
%%% End:
